\begin{table}[h]
\renewcommand\thetable{1}   % forcing the specific number
\captionsetup{font=footnotesize}  % smaller caption
\vspace*{10pt}
\centering

\begin{tabular}{l c c c}
\toprule
Model Name & Acc. Laptop & Acc. Restaurant & Release \\
\midrule
LCF-ATEPC & 82.29 & 90.18 & Jan 2020 \\
BERT-ADA & 80.23 & 87.89 & Nov 2019 \\
BAT & 79.35 & 86.03 & Feb 2020 \\
\bottomrule
\end{tabular}

\caption{
State of the art results on the most common evaluation dataset (SemEval 2014 Task 4 SubTask 2).
The second model was presented in the previous section. All use BERT as the language model.
}
\label{tab:semeval}
\end{table}



\begin{table}[h]
\renewcommand\thetable{2}
\captionsetup{font=footnotesize}
\vspace*{10pt}
\centering

\begin{tabular}{l c c}
\toprule
Test Name & Acc. Laptop & Acc. Restaurant \\
\midrule
Semeval & 79.23 & 85.17 \\
\midrule
Test A: Course-Grained   & 71.63 & 75.62 \\
Test B: Aspect Condition & 35.94 & 40.54 \\
Test C: Add Emotional Sentence  & & \\
\quad \qquad \: a) Process pos/neu (add neg) & 66.42 (-13.58\%) & 83.68 (-4.07\%)  \\
\quad \qquad \: b) Process neg/neu (add pos) & 51.47 (-26.15\%) & 55.34 (-12.52\%) \\
\bottomrule
\end{tabular}

\caption{
The test results of the \texttt{classifier-rest-0.2}
and the \texttt{classifier-lapt-0.2} models (the BERT-ADA architecture).
}
\label{tab:model-tests}
\end{table}



\begin{table}[h]
\renewcommand\thetable{3}
\captionsetup{font=footnotesize}
\vspace*{10pt}
\centering

\begin{tabular}{l l l}
\toprule
Test Name & Acc. Laptop & Acc. Restaurant \\
\midrule
Semeval & 79.00 (-0.98\%)   & 83.13 (-2.41\%)   \\
Test B  & 74.41 (+106.51\%) & 79.55 (+108.81\%) \\
\bottomrule
\end{tabular}

\caption{
The test results of the \texttt{classifier-rest-0.2} and the \texttt{classifier-lapt-0.2} models
enhanced by the basic reference recognizer (the version 0.1).
The recognizer parameters chosen based on the training data.
The improvement of the model performance is given in brackets.
}
\label{tab:aux-model-reference}
\end{table}



\begin{table}[h]
\renewcommand\thetable{4}
\captionsetup{font=footnotesize}
\vspace*{10pt}
\centering

\begin{tabular}{
    >{\raggedright}p{3cm}
    >{\centering}p{1.2cm}
    >{\centering}p{1.2cm}
    >{\centering}p{1.1cm}
    >{\centering\arraybackslash}p{1.1cm}
}\toprule
\multirow{3}{*}{Pattern Recognizer} & \multicolumn{2}{c}{Acc. Laptop} & \multicolumn{2}{c}{Acc. Restaurant} \\
 & Train  & Test   & Train  & Test \\
 & 10.33\% & 23.82\% & 16.88\% & 26.43\% \\
\midrule
Random         &  6.28 & 18.42 &  8.39 & 11.15 \\
Attention      & 36.82 & 44.08 & 29.93 & 32.77 \\
Gradient       & 14.64 & 16.45 & 24.84 & 26.69 \\
\textbf{Basic} & 53.14 & 55.26 & 73.52 & 62.16 \\
\bottomrule
\end{tabular}

\caption{
The key token recognition based on an explanation provided by a pattern recognizer.
Evaluated on examples that have at least one key token.
The percents under the dataset names describe the amount of examples that contain a key token within each dataset.
}
\label{tab:recognition-key-token}
\end{table}


\begin{table}[h]
\renewcommand\thetable{5}
\captionsetup{font=footnotesize}
\vspace*{10pt}
\centering

\begin{tabular}{l c c}
\toprule
\multirow{2}{*}{Pattern Recognizer} & Acc. Laptop & Acc. Restaurant \\
                                    & 41.07\% & 41.52\% \\
\midrule
Random         & 14.89 & 11.61 \\
Attention      & 40.46 & 36.56 \\
Gradient       & 17.94 & 21.94 \\
\textbf{Basic} & 50.00 & 49.25 \\
\bottomrule
\end{tabular}

\caption{
The recognition of the key pair of tokens based on an explanation provided by a pattern recognizer.
Evaluated on examples that have at least one key token pair.
The percents under the dataset names describe the amount of examples that contain a key token pair within each dataset.
}
\label{tab:recognition-key-token-pair}
\end{table}


